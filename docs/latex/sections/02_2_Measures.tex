\subsection{Knochenmaße}

Jeder Knochen kann auf viele unterschiedliche Weisen vermessen werden, beispielsweise kann man den Umfang an einer bestimmten Stelle bestimmen, die Länge an der längsten Stelle oder die unterschiedlichen Breiten an den jeweiligen Enden oder in der Mitte. 
Diese verschiedenen »Dimensionen« wurden versucht, möglichst eindeutig zu definieren und damit zu standardisieren, doch \textit{»There will always be discrepancies from one research worker to the next which woll influence the final results.«} \cite{Levine1981}
Im Datensatz wurden die unterschiedlichen Mess-Dimensionen durch das Attribut \texttt{MEASTYPE} gekennzeichnet, die verwendeten Codes beziehen sich größtenteils auf die Arbeit von \cite{Levine1981}.
Wichtig ist an dieser Stelle vor allem zu beachten, dass auch die Maße für einen Vergleich nicht alle in einen Topf geworfen werden dürfen: Vermischt man Längenmaße mit Breite oder Umfang, verfälscht das das Ergebnis enorm.
Der Versuch, wenigstens unterschiedliche Dimensionen der gleichen Obergruppe (Länge, Breite, Umfang, Höhe) zusammenzufassen, wurde in \texttt{an\_9} und \texttt{an\_10} unternommen, doch aufgrund der Vielzahl und Komplexität der unterschiedlichen Messarten (und aufgrund des nicht vorhandenen Vorwissens der Autorin über Knochen, deren Bezeichnungen und Eigenarten) aufgegeben. 
Stattdessen wurden für jeden Knochentyp die vier Dimensionen mit den meisten Messwerten zur Weiterverarbeitung ausgewählt (s. \texttt{an\_11})